\section{Pricing of individual objects}

\begin{frame}
  \frametitle{Inspiration}

  \begin{center}
    \includegraphics[width=0.8\linewidth,keepaspectratio,frame]{ime-2017-insurance-of-things} \\
    \link{https://github.com/vigou3/ime-2017-insurance-of-things/releases/tag/v1.0}{Slides
      available on GitHub}
  \end{center}
\end{frame}

\begin{frame}
  \frametitle{Different types of peril}

  \begin{assumption}
    Claims due to three main types of peril:
    \begin{enumerate}
    \item accidental damage, theft, vandalism; \alert{few} items affected ($<$ 10 on average)
    \item fire; \alert{100\%} of items affected
    \item flooding, water damage, sewer backup; \alert{25\%} of items
      affected
    \end{enumerate}
  \end{assumption}
  \pause

  \begin{consequence}
    Aggregate claim amount for insured $i = 1, \dots, n$ for one period:
    \begin{equation*}
      S_i = S_i^\bris + S_i^\feu + S_i^\eau
    \end{equation*}
  \end{consequence}
\end{frame}

\begin{frame}
  \frametitle{Dependence -- Step I}

  \begin{assumption}
    Aggregate claim amount for a given type of peril is compound Poisson:
    \begin{equation*}
      S_i^\DOT = X_{i1}^\DOT + \dots + X_{iN_i^\DOT}^\DOT
      \sim \text{Compound Poisson}(\lambda^\DOT, F^\DOT)
    \end{equation*}
  \end{assumption}
  \pause

  \begin{consequence}
    Aggregate claim amount for insured $i$ is compound Poisson:
    \begin{equation*}
      S_i = S_i^\bris + S_i^\feu + S_i^\eau \sim
      \text{Compound Poisson}(\lambda, F),
    \end{equation*}
    with
    \begin{align*}
      \lambda &= \lambda^\bris + \lambda^\feu + \lambda^\eau \\
      F(x) &= \frac{\lambda^\bris F^\bris(x) + \lambda^\feu F^\feu(x) +
               \lambda^\eau F^\eau(x)}{\lambda}
      \end{align*}
  \end{consequence}
\end{frame}

\begin{frame}
  \frametitle{Dependence -- Step II}

  \begin{assumption}
    Amount of single claim is
    \begin{equation*}
      X_{ij}^\DOT = V_1 + ... + V_{M_{ij}^\DOT},
    \end{equation*}
    where
    \begin{itemize}
    \item $M^\DOT$ is the number of items affected (strictly
      positive discrete distribution)
    \item $V$ is the value of an item (i.i.d.)
    \end{itemize}
  \end{assumption}
  \pause

  \begin{consequence}
    $X_{ij}^\DOT$ is a compound zero-truncated
  \end{consequence}
\end{frame}

\begin{frame}
  \frametitle{Parametrization}

  We need to determine the following elements from judgment, expert
  opinion or collateral data:
  \begin{itemize}
  \item Poisson parameters $\lambda^\bris$, $\lambda^\feu$ and
    $\lambda^\eau$ for the number of claims by type of peril
  \item Zero-truncated distributions of the number of items affected
    per type of peril
  \item Market value of an item
  \end{itemize}
\end{frame}

\begin{frame}
  \frametitle{Pricing}

  \begin{itemize}
  \item Average aggregate claim amount in portfolio:
    \begin{equation*}
      W = \frac{S_1 + \dots + S_n}{n}
    \end{equation*}
  \item<2-> Pure premium: $\esp{W}$
  \item<3-> Premium with safety loading:
    \begin{equation*}
      \pi = \TVaR_\alpha(W) = \esp{W|W > \VaR_\alpha(W)}
    \end{equation*}
  \end{itemize}
\end{frame}


%%% Local Variables:
%%% mode: latex
%%% TeX-engine: xetex
%%% TeX-master: "ime-2017-workshop-computational-actuarial-science-r"
%%% End:
